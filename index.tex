\documentclass[14pt]{article}
\usepackage[left=2cm, right=2cm, top=2cm]{geometry}
\usepackage{amsmath}
\usepackage{enumitem}
\usepackage{tabto}
\begin{document}

%\hspace{10cm}\text{Name:________________________}
%\textbf{Name :                        }\hspace*{12cm}
%\textbf{Roll No :                        }\\
\begin{center}
\large{\textbf{Government College of Engineering Kannur\\}}
\large{\textbf{Department of Computer Science and Engineering\\}}
\large{\textbf{CSD334: Mini-Project\\}}
\end{center}

\begin{center}
\small{\textbf{\underline{ Project Proposal}\\}}
\end{center}

\begin{table}[ht!]
\begin{center}
\begin{tabular}{|cccc|}
\hline
\multicolumn{4}{|c|}{\textbf{Proposal Prepared By}}                        \\ \hline
\multicolumn{1}{|c|}{Sl No.} & \multicolumn{1}{c|}{KTU Reg No.} & \multicolumn{1}{c|}{Name of Student}              & Signature of Student \\ \hline
\multicolumn{1}{|c|}{1}      & \multicolumn{1}{c|}{KNR22CS08}   & \multicolumn{1}{c|}{Amarnath Panneri} &           \\ \hline
\multicolumn{1}{|c|}{2} & \multicolumn{1}{c|}{KNR22CS043} & \multicolumn{1}{c|}{Muhammed Hanoon Zameel} &  \\ \hline
\multicolumn{1}{|c|}{3} & \multicolumn{1}{c|}{KNR22CS062} & \multicolumn{1}{c|}{Swalih.T} &  \\ \hline
\multicolumn{1}{|c|}{4} & \multicolumn{1}{c|}{KNR22CS065} & \multicolumn{1}{c|}{Tom Sebastain} &  \\ \hline
\end{tabular}
\end{center}
\end{table}


\begin{enumerate}
\item Project Title : \textbf{Lab Exam Management System}
\item Problem Statement \\
\tabto{.5cm}Existing lost and found management systems suffer from fragmentation and
inefficiency due to their reliance on outdated methods such as physical notice
boards and disparate online platforms. This lack of a centralized and accessible
platform hinders the timely and effective reporting of lost items by users and found
items by finders. Consequently, individuals experiencing loss and those who have
found items encounter difficulties in navigating the process, resulting in prolonged
periods of uncertainty and decreased chances of successful item recovery.\\
\tabto{.5cm} Our project aims to address the inefficiencies and challenges of traditional lost and found systems by developing a user-friendly, single-page web application designed to manage lost and found items.
\item Solution approach \\
\tabto{.5cm}Our proposed exam evaluation system will streamline the examination process, from creation to evaluation, ensuring accuracy and security. Below are the key features and functionalities.
\begin{enumerate}
\item \textbf{User Registration and Authentication}
\begin{itemize}
\item Users can join as a student or teacher by creating accounts or logging in using existing credentials.
\item Authentication ensures secure access to the system.
\end{itemize}
\item \textbf{Room Creation and Exam Setting}
\begin{itemize}
\item Examiners can create exam rooms and upload question papers.
\item Features include specifying exam dates and durations.
\end{itemize}
\item \textbf{Platform-Independent compiler and Proctored UI}
\begin{itemize}
\item The system includes a platform-independent compiler interface for seamless access on any device.
\item A proctored UI prevents students from copying or switching tabs during the examination.
\end{itemize}
\item \textbf{AI Partial Output Detection}
\begin{itemize}
\item An AI-powered module evaluates student's code for partial correctness.
\item This ensures fair grading even if the solution is incomplete but correct to some extent.
\end{itemize}
\item \textbf{Student-Wise Report Generation}
\begin{itemize}
\item The system generates detailed results for each student.
\item Results can be downloaded in PDF format for easy sharing and record-keeping.
\end{itemize}
\item \textbf{Centralized Database}
\begin{itemize}
\item A centralized database stores information about exams, subjects, dates, and student reports.
\item MySQL or a similar relational database management system (RDBMS) is used for efficient data management.
\end{itemize}
\item \textbf{User Notifications}
\begin{itemize}
\item Automated notifications inform students about exam schedules and results.
\item Alerts are sent via email or SMS for timely updates.
\end{itemize}
\end{enumerate}


\item Does any other solution already exist for the stated problem? If yes, give details.
\begin{itemize}
\item Yes, there are existing solutions for exam evaluation, but they often suffer from various limitations.
\begin{enumerate}
\item Manually Conducted Exams
\begin{itemize}
\item {Limitations}
\begin{itemize}
\item Time-Consuming: Teachers must manually check each system for output verification, which is inefficient.  
\item No Proper Partial Output Verification: Results are assessed as correct or incorrect, without acknowledging partial correctness.  
\item System Variability: Each system may behave differently due to inconsistent configurations, causing discrepancies in output.  
\end{itemize}
\end{itemize}
\item Online Exam Platforms
\begin{itemize}
\item HackerRank
\item {Limitations}
\begin{itemize}
\item No Partial Output Detection: Unable to evaluate partially correct solutions, which is essential for fair assessment in coding exams.  
\item Not Best for College Lab Exams: The platform is better suited for corporate hiring assessments than for educational settings.  
\item Limited Offline Support: Requires constant internet connectivity, which can be a challenge for some labs.  
\end{itemize}
\item HackerEarth
\item {Limitations}
\begin{itemize}
\item High Cost: Not affordable for all educational institutions.  
\item Complex Setup: Requires significant technical expertise for exam creation and management.  
\end{itemize}
\item Other Online Platforms
\item {Limitations}
\begin{itemize}
\item Lack of AI Features: Many do not offer advanced features like AI-based partial output evaluation.  
\item Limited Customization: Few platforms allow fully tailored exams to suit specific academic needs.  
\item Proctoring Gaps: Some platforms have limited proctoring features, increasing the risk of cheating.  
\end{itemize}
\end{itemize}
\end{enumerate}
\item A centralized, AI-powered exam evaluation system with robust proctoring, partial output detection, and seamless integration for college labs can effectively address these limitations.
\end{itemize}


\item In what way is your proposed solution better than the existing solution(s)?\\
\tabto{.5cm}Our proposed solution for lost and found management offers several advantages
over existing methods.
\begin{enumerate}
\item \textbf{Centralized Platform}
\begin{itemize}
\item Unlike physical notice boards and fragmented online platforms, our single-page web application provides a centralized platform.
\item Users can report lost items and finders can report found items from one location, streamlining the process.
\end{itemize}
\item \textbf{Accessibility}
\begin{itemize}
\item Our web application is accessible to anyone with an internet connection.
\item No need for physical presence or multiple app downloads, increasing the
chances of reuniting lost items with owners.
\end{itemize}
\item \textbf{User-Friendly Interface}
\begin{itemize}
\item We prioritize a simple, intuitive interface for reporting lost and found
items. 
\item Features like easy image uploads enhance user experience and encourage
participation.
\end{itemize}
\item \textbf{Comprehensive Information}
\begin{itemize}
\item Users can provide detailed descriptions and images of lost items.
\item Finders report items through provided contact information, facilitating
swift communication.
\end{itemize}
\item \textbf{Efficiency}
\begin{itemize}
\item Streamlined reporting and centralized management reduce time and effort for users and administrators. 
\item Quicker responses and higher recovery rates compared to fragmented
solutions.
\end{itemize}
\end{enumerate}
\item List out the stakeholders of the proposed project.
\begin{enumerate}
\item \textbf{Users}
\begin{itemize}
\item Individuals who report lost items or find items.
\item Interact directly with the application.
\item Search for lost items and provide feedback.
\end{itemize}
\item \textbf{Administrators}
\begin{itemize}
\item Responsible for managing the application. 
\item Moderate content and ensure smooth operation.
\item Handle user inquiries and maintain system integrity. 
\end{itemize}
\item \textbf{Community/Institution}
\begin{itemize}
\item Organizations where the application is deployed
\item Provide support, resources, or endorsement.
\end{itemize}
\end{enumerate}
\item What will be the impact of this project work on society directly or indirectly?\\
\tabto{.5cm}The impact of a lost and found website on society can be significant, both directly
and indirectly.
\begin{enumerate}
\item \textbf{Direct Impact}
\begin{itemize}
\item \textbf{Efficiency} : The website can help individuals quickly locate lost items,
reducing the time and effort spent on searching through traditional methods like calling businesses or physically visiting lost and found offices.
\item \textbf{Convenience} : Users can report lost items and check for found items at any
time from the comfort of their homes, making the process more convenient
and accessible.
\item \textbf{Emotional Relief} : Losing valuable items can be distressing. The ability to
easily search for lost items and potentially have them returned can provide emotional relief and reduce stress for individuals
\end{itemize}
\item \textbf{Indirect Impact}
\begin{itemize}
\item \textbf{Reduced Waste} : By facilitating the return of lost items, the website can
help reduce unnecessary waste. Instead of replacing lost items, individuals can reclaim them, leading to less consumption and a smaller environmental
footprint.
\item \textbf{Economic Benefits} : Avoiding the need to replace lost items can save
individuals money, leading to potential economic benefits for both
individuals and businesses.
\item \textbf{Safety} : Returning lost items promptly can contribute to public safety by
preventing potential hazards or accidents associated with misplaced items, such as lost identification documents or personal belongings.
\end{itemize}
\end{enumerate}
\tabto{.5cm} Overall, the project's impact on society can enhance efficiency, convenience, and
community engagement while indirectly contributing to environmental sustainability, economic savings, safety, and trust within the community.
\item What technologies do you plan to use for the front-end and back-end?
\begin{enumerate}
\item \textbf{Front-end (Client-side)} :
\begin{itemize}
\item \textbf{React.js} : A JavaScript library for building user interfaces. It allows developers
to create reusable UI components and manage the state of the application
efficiently
\end{itemize}
\item \textbf{Back-end (Server-side)} :
\begin{itemize}
\item \textbf{Node.js} : A JavaScript runtime environment that allows developers to run
JavaScript code on the server-side. It provides an event-driven architecture that is well-suited for building scalable and high-performance server applications.
\item \textbf{Express.js} : A web application framework for Node.js. It provides a robust set
of features for building web servers and APIs, including middleware support,
routing, and HTTP utilities
\end{itemize}
\item \textbf{Database} :
\begin{itemize}
\item \textbf{MongoDB} : A NoSQL database that stores data in a flexible, JSON-like format.
It is schema-less, allowing developers to store and retrieve data without a predefined schema. MongoDB is commonly used in MERN stack applications due to its flexibility and scalability.
\end{itemize}
\tabto{.5cm} Together, these technologies form the MERN stack, which provides a comprehensive solution for building our web applications with a JavaScript-based
technology stack
\end{enumerate}

\end{enumerate}
\begin{table}[ht!]
\begin{center}
\begin{tabular}{|cll|}
\hline
\multicolumn{3}{|c|}{\textbf{Proposal Approved By}}                                                                   \\ \hline
\multicolumn{1}{|c|}{Name of Guide} & \multicolumn{1}{c|}{Signature of Guide} & \multicolumn{1}{c|}{Date of Approval} \\ \hline
\multicolumn{1}{|l|}{Dr. Rafeeque P C}              & \multicolumn{1}{l|}{}                   &                                       \\ \hline
\end{tabular}
\end{center}
\end{table}
\end{document}

